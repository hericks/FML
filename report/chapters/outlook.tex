\chapter{Outlook}

\section[Experience]{Experience \hfill \small \normalfont\textit{by Alessandro Motta}}
In this project we decided to rely on linear function approximation. It was difficult trying to stick to a certain approach since reinforcement learning (RL) is such a vast subject. Nevertheless we had to do it, because we knew we would not have enought time to try out everything. The tasks that our agent was given were only a part of the goal for us in this project. The other part was to really be able to have a deeper understanding of the methods we used. That is why we decided not to use RL libraries that are already implemented in Python. We never had the problem that we did not have any ideas of what we wanted to do when we noticed that our agent was not learning. It was a shame that we were not able to implement all the ideas due to the time constraint. In this section, we will further describe what we would like to implement in the future without a deadline.
\subsubsection*{Implementing a better genetic crossover for our features}
In our linear method we implemented genetic crossover for the features, we observed it could work on the coin collecting agent but did not manage to make it work well for the crate agent. Making it work for the crate agent would have had a huge impact on our workflow. Also because right now we do not know if our agent is not learnign ideally because we don't have the right features or because we did not combine them the right way. 
\subsubsection*{Non linear approximation function}
Using a linear method made us concentrate a lot on feature design, which we enjoyed, yet we would have liked to learn more about the other non linear methods and to descover the differences and the opportunities that these methods could offer us. For example implementing non binary features. Something we would have also liked to do, is change to off policy learnig with Batch Descent methods.
\subsubsection*{Training against other agents}
Due to the fact that we did not manage to create an ideal crate agent we decided not to train against other agents. This would have not made a lot of sense since we did not manage to make him drop bombs on a certain target with our current reward and feature system. In the future we would like to train him for this task. But we will see how he performs in the tournament, maybe he is not that bad in playing against others after all.

\section{Possible improvements of the environment}

First of all, we want to stress that the Bomberman environment worked really well out of the box. The setup was easy and the initial hooks and templates guaranteed an excellent start. Still, we have some ideas how the environment and overall project can be improved!

\begin{itemize}
	\item \textbf{Python Training API}. One of the major drawbacks of the environment was its inability to train agents via a Python API. When wanting to train an agent the basic setup requires a command of the following form
	
		\begin{center}
			\texttt{python main.py play --agents my\_agent --train 1 --n-rounds 100 --no-gui}	
		\end{center}
	
		It is not possible to pass custom parameters to the agent. For this we found multiple solutions. One being to set custom parameters as environment variables. The agent \texttt{my\_agent} then reads the environment variables during its setup. Unfortunately, this approach does not work to pass increasingly complex parameters, e.g. Python objects, to the agent. This was necessary for the genetic feature extraction algorithm we implemented. This algorithm needed to start training sessions independently with different complex (boolean expression trees) as parameters for the agent. Our rather inelegant solution for this problem was to save the expression tree in a \texttt{pickle}-file. The agent to train subsequently read this file. Even though this workflow works, it is quite inefficient. An official API with function calls like
	
		\begin{center}
			\texttt{RL.train(agent\_name, num\_rounds, args)},
		\end{center}
	
		where \texttt{args} are passed to the agent in its setup function would be an optimal solution. 
	
	\item \textbf{Extra state transition}. We noticed that \texttt{game\_events\_occurred} is called once before the game starts with \texttt{old\_game\_state} as well as \texttt{self\_action} set to \texttt{None} and \texttt{new\_game\_state} being the initial state of the game. From further discussions with other teams, we noticed that this behaviour created some confusion. It should be mentioned in the description of the repository. \\
	
	\item \textbf{Missing state transitions.} Another closely related possible improvement would be to include a call \texttt{game\_events\_occurred} for the last state transition. This is currently not done, since learning algorithms don't rely on the \emph{very} last state of the environment. However, as some points we wanted to log multiple properties of the very last state, e.g. the number of coins/crates left on the board. This is currently not possible, because the last state is not even included in the call of \texttt{end\_of\_round}. 
\end{itemize}

Overall, the setup was excellent to work with and our team had a great (reinforcement) learning experience \texttt{;)}. We are eager to work with reinforcement learning in again some time in the future and are excited to see the final results of the other teams. 